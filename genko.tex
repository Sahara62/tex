\documentclass[twocolumn]{jsarticle}
\begin{document}

\title{UPPAALを用いた自動運転車の\\群制御アルゴリズムのモデル化と検証}
\author{佐原優衣}
\maketitle
\section*{はじめに}
UPPAALを用いた自動運転車の群制御アルゴリズムのモデル化について中村研究室 所属 佐原が説明します。
\section*{背景・目的}
本研究の背景といたしましては,近年,自動運転技術が発達しており,生活に使われるようになる日も遠くはないと考えられます。多数の自動運転車が実際に走行する都市空間において,個々の車が,他車を考慮しない経路選択を自律的に行うと,問題が発生する可能性があります。したがって,自動運転車の群制御アルゴリズムが必要となります。

本研究の目的は自動運転車の群制御アルゴリズムを形式的に記述し,その性質を形式的に検証する手法を提案することです。
\section*{方法}
方法の説明をしていきます。群制御アルゴリズムをオートマトンで記述し,モデル検査を用いて次の性質について検証を行います。デッドロックに陥らない性質と,目的地に時間内に到着する性質を検証するために,時間オートマトンによる時間制約性質の記述ができるUPPAALを採用します。
\newpage
\section*{例題}
UPPAALによるモデル化と検証について例題を使って説明します。すれ違えないほど狭い一本道で車が個々に進んだ場合にデッドロックが生じるかどうかを検証していきます。群制御アルゴリズムがない場合を検査すると,この図のようにデッドロックが生じます。このように,UPPAALは検証内容が満たされない時,反例を出力します。
次に集中制御として,信号機オートマトンを導入した場合を検証します。信号機オートマトンにより,道の方向が定められ,車が制御されるので,デッドロックが生じません。
\section*{まとめ}
今回は小さい例題についてUPPAALを用いてモデル化と検証を行った。
今後の課題として,実際の規模の都市空間のモデル化を行い,その街で数千台の自動運転車の群制御アルゴリズムをUPPAALでモデル化し,検証を行えるか検討いたします。また,UPPAALでどの程度群制御アルゴリズムをモデル化を行うかについても検討すべきと考えています。
\end{document}